\documentclass[a4paper,11pt]{article}
\usepackage[utf8]{inputenc}
\usepackage[T1]{fontenc}
\usepackage[english]{babel}
\usepackage{geometry}
\usepackage{hyperref}
\usepackage{amsmath}
\usepackage{amssymb}
\usepackage{listings}
\usepackage{graphicx}
\usepackage{xcolor}

\geometry{left=3cm,right=3cm,top=2.5cm,bottom=2.5cm}

\hypersetup{
	colorlinks=true,
	linkcolor=blue,
	urlcolor=cyan,
	pdftitle={Specification: Joint Angle Calculation for Robot Arm},
	pdfauthor={Eduard Burghardt}
}

\title{Specification: Joint Angle Calculation for Robot Arm}
\author{Eduard Burghardt}
\date{\today}

% Define a macro for missing sections (visible but compile-able)
\newcommand{\missing}[1]{\textbf{\textcolor{red}{[MISSING: #1]}}}

\begin{document}
	
	\maketitle
	
	\tableofcontents
	\newpage
	
	\section{Introduction}
	
	This document specifies a small program to calculate joint angles of a multi-segment robot arm.  
	Visualization and user interface are not part of this project.
	
	\section{Objectives}
	
	The program shall read a configuration file describing the robot arm’s structure and  
	calculate joint angles to reach a given target position. The results shall be output to a file.
	
	\section{Requirements}
	
	\subsection{Functional Requirements}
	
	\begin{itemize}
		\item Read a JSON configuration file including:
		\begin{itemize}
			\item Number of joints
			\item Lengths of arm segments
			\item Direction of arm segments
			\item Rotation axes of joints (each with maximal 2 degrees of freedom)
			\item Minimum and maximum joint angle limits
			\item Target position in the global coordinate system
		\end{itemize}
		\item Calculate joint angles (inverse kinematics) to reach the target position.
		\item Output the calculated angles to a file (e.g., JSON or plain text).
		\item The global coordinate system origin is located at the first joint.
		\item Handle errors such as unreachable targets gracefully.
		\item \missing{Additional motion constraints, tool orientation, or obstacle avoidance logic.}
	\end{itemize}
	
	\subsection{Additional Functional Requirements}
	
	\begin{itemize}
		\item Each joint should have a defined position in the global coordinate system.
		\item The definition of the robotic arm should include a direction vector.
		\item Each joint's rotation is limited to a maximum of two degrees of freedom (DOF).
	\end{itemize}
	
	\subsection{Non-Functional Requirements}
	
	\begin{itemize}
		\item Platform independence: development on Mac OS X, execution on Windows.
		\item Usability: user has minimal computer skills, can only edit the JSON file if needed.
		\item Performance: calculations should complete in reasonable time; no real-time constraints.
		\item Extensibility: design should allow future integration of visualization and GUI.
		\item No current considerations are required for privacy or security measures.
	\end{itemize}
	
	\section{System Architecture}
	
	\begin{itemize}
		\item \textbf{Parser Module}: reads and validates the JSON configuration.
		\item \textbf{Kinematics Module}: performs inverse kinematics calculation (e.g., CCD algorithm).
		\item \textbf{Output Module}: writes the computed joint angles to a file.
		\item \missing{Description of module communication and error handling}
	\end{itemize}
	
	\section{Data Format}
	
	\subsection{JSON Configuration File}
	
	\begin{verbatim}
		{
			"joints": [
			{
				"name": "base",
				"dir" : [0, 0, 1],
				"length": 100.0,
				"rotation_axis": [0, 0, 1],
				"min_angle": -90,
				"max_angle": 90
			},
			{
				"name": "elbow",
				"dir" : [0, 1, 0],
				"length": 80.0,
				"rotation_axis": [0, 1, 0],
				"min_angle": 0,
				"max_angle": 135
			}
			],
			"target": [120, 80, 0]
		}
	\end{verbatim}
	
	\subsection{Output File}
	
	Example output (JSON):
	
	\begin{verbatim}
		{
			"joint_angles": {
				"base": 45.0,
				"elbow": 30.0
			}
		}
	\end{verbatim}
	
	\section{Mathematical Model}
	
	\subsection{Kinematics}
	
	\begin{itemize}
		\item Each joint rotation is modeled as a rotation around a fixed axis by an angle $\theta_i$.
		\item The end-effector position is computed by sequential transformations of all joints in the global coordinate frame.
		\item The inverse kinematics finds angles $\theta_i$ that position the end-effector as close as possible to the target.
	\end{itemize}
	
	\subsection{Constraints}
	
	\begin{itemize}
		\item Rotations are limited to one degree of freedom per joint (1 DOF).
		\item Joint angle limits are enforced.
		\item If the target is unreachable, the best possible approximation is computed.
		\item \missing{Further mathematical details such as algorithm choice and solution approach}
	\end{itemize}
	
	\section{Error Handling}
	
	\begin{itemize}
		\item Invalid or incomplete JSON file: produce an error message and terminate.
		\item Unreachable target: output a warning and the closest achievable solution.
		\item Joint angles exceeding limits are clamped to valid ranges.
		\item \missing{Logging mechanisms and detailed error handling strategy}
	\end{itemize}
	
	\section{Appendix}
	
	\missing{Use cases, glossary, test cases, and example outputs}
	
\end{document}
