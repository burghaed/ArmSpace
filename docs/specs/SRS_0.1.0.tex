\documentclass[a4paper,11pt]{article}
\usepackage[utf8]{inputenc}
\usepackage[T1]{fontenc}
\usepackage[english]{babel}
\usepackage{geometry}
\usepackage{hyperref}
\usepackage{amsmath}
\usepackage{amssymb}
\usepackage{listings}
\usepackage{graphicx}
\usepackage{xcolor}

\geometry{left=3cm,right=3cm,top=2.5cm,bottom=2.5cm}

\title{\vspace{-2cm}Software Requirements Specification\\for ArmSpace Project}
\author{Author: [Your Name]}
\date{\today}

\begin{document}
	\maketitle
	
	\tableofcontents
	\newpage
	
	\section{Introduction}
	\subsection{Purpose}
	This document defines the software requirements specification (SRS) for the ArmSpace project. It outlines the functionality, interfaces, constraints, and testing strategies required for the initial version of the system.
	
	\subsection{Scope}
	ArmSpace is a software tool designed to process structured input data, perform calculations via a computation engine, and (in future versions) provide GUI and visualization capabilities. This version focuses on core data parsing and computational correctness.
	
	\subsection{Definitions and Abbreviations}
	\begin{itemize}
		\item \textbf{SRS}: Software Requirements Specification
		\item \textbf{JSON}: JavaScript Object Notation
		\item \textbf{gtest}: Google Test framework
	\end{itemize}
	
	\section{Overall Description}
	\subsection{Product Perspective}
	ArmSpace is a standalone tool aimed at data processing and computational modeling, with possible future expansion to GUI-based interactions and visual output.
	
	\subsection{User Characteristics}
	Users are expected to be engineers or developers familiar with structured data and mathematical modeling.
	
	\subsection{Constraints}
	\begin{itemize}
		\item Implementation in C++
		\item Input and output in JSON format
		\item Command-line based interface
	\end{itemize}
	
	\section{Functional Requirements}
	\begin{itemize}
		\item FR1: The system shall parse structured JSON input.
		\item FR2: The system shall execute computations based on parsed input.
		\item FR3: The system shall output structured results in JSON format.
	\end{itemize}
	
	\section{Non-Functional Requirements}
	\begin{itemize}
		\item NFR1: The system shall process typical input files in under 2 seconds.
		\item NFR2: The system shall produce human-readable output JSON files.
		\item NFR3: The system shall be unit tested with minimum 90\% code coverage.
	\end{itemize}
	
	\section{Testing Strategy}
	\subsection{Overview}
	This section outlines the testing strategy for the ArmSpace project. The primary goal of testing is to ensure functional correctness, stability, and maintainability of all system components. Testing will include unit tests, regression tests, and optionally integration tests.
	
	\subsection{Unit Testing}
	Unit tests will be implemented using the \texttt{GoogleTest (gtest)} framework. Each major component (e.g., the parser, calculation engine) will be tested in isolation. Unit tests will verify the behavior of public interfaces, handle edge cases, and confirm that invalid inputs are handled gracefully.
	
	\subsection{Regression Testing}
	Regression tests will be based on pre-prepared input and reference output JSON files. These tests will:
	\begin{itemize}
		\item Load an input JSON test case.
		\item Run the relevant module (e.g., calculation engine).
		\item Compare the actual output with a known-good reference file.
	\end{itemize}
	
	These tests ensure that code changes do not introduce unintended side effects or break existing functionality. A parameterized test setup using \texttt{gtest} will be used to run multiple test cases efficiently.
	
	\subsection{Integration Testing}
	In this context, regression tests will also serve as integration tests, since they validate the cooperation of multiple modules (e.g., parser + calculator). Further integration testing may be developed if system complexity increases.
	
	\subsection{Automation and CI}
	The test suite will be integrated into a continuous integration (CI) pipeline (e.g., GitHub Actions, GitLab CI) to automatically run all unit and regression tests on code commits and pull requests.
	
	\section{Appendix}
	\subsection{References}
	\begin{itemize}
		\item GoogleTest Framework: \url{https://github.com/google/googletest}
		\item IEEE 830-1998: Recommended Practice for Software Requirements Specifications
	\end{itemize}
	
\end{document}
