\documentclass{article}
\usepackage{float}
\usepackage{hyperref}
\usepackage{geometry}
\usepackage{amsmath}
\usepackage{graphicx}
\geometry{a4paper, margin=1in}

\title{Software Requirements Specification (SRS) \\ ArmSpace Project}
\author{Your Name}
\date{June 14, 2025}

\begin{document}

\maketitle

\tableofcontents

\section{Introduction}
\subsection{Purpose}
This document defines the software requirements specification (SRS) for the \textbf{ArmSpace} project. It outlines the functionality, interfaces, constraints, and testing strategies required for the initial version of the system.

\subsection{Scope}
ArmSpace is a software tool designed to process structured input data, perform calculations via a computation engine, and provide motor control commands in JSON format.

\subsection{Definitions and Abbreviations}
\begin{itemize}
    \item \textbf{SRS}: Software Requirements Specification
    \item \textbf{JSON}: JavaScript Object Notation
    \item \textbf{CMake}: Cross-platform build system
    \item \textbf{gtest}: Google Test framework
    \item \textbf{Quaternion}: A mathematical representation for rotation, avoiding gimbal lock.
\end{itemize}

\section{Overall Description}
\subsection{System Architecture Overview}
The system consists of multiple **modules** working together:

\begin{itemize}
    \item \textbf{Input Parser}: Validates and processes JSON input files.
    \item \textbf{Computation Engine}: Handles quaternion-based calculations and motor tick conversion.
    \item \textbf{Logging System}: Uses \textbf{spdlog} for structured error reporting.
    \item \textbf{Output Generator}: Produces JSON-formatted motor control values.
\end{itemize}

\subsection{Architecture Diagram}
\begin{figure}[h]
    \centering
    \includegraphics[width=0.8\textwidth]{architecture_diagram.png}
    \caption{ArmSpace System Architecture}
\end{figure}

\subsection{Supported Platforms}
ArmSpace is designed to run on:
\begin{itemize}
    \item **Linux**
    \item **macOS**
    \item **Windows**
\end{itemize}
Future versions may be ported to additional platforms.

\subsection{Constraints}
\begin{itemize}
    \item Implemented in C++
    \item Uses \textbf{nlohmann/json} for JSON parsing and writing
    \item Uses \textbf{spdlog} for logging
    \item Uses \textbf{CMake} as the build system for cross-compiling to Windows
    \item Input and output in JSON format
    \item Command-line interface (CLI) for the current version
\end{itemize}

\section{Project Folder Structure}
The following **organized directory layout** ensures clarity and maintainability:

\begin{verbatim}[fontsize=\small]
ArmSpace/
|-- CMakeLists.txt
|-- cli/
|   `-- CMakeLists.txt
|-- libs/
|   |-- error/
|   |   `-- CMakeLists.txt
|   |-- parser/
|   |   `-- CMakeLists.txt
|   |-- writer/
|   |   `-- CMakeLists.txt
|   |-- engine/
|   |   `-- CMakeLists.txt
|   `-- validator/
|       `-- CMakeLists.txt
|-- tests/
|   `-- CMakeLists.txt
`-- build/
\end{verbatim}

\section{JSON Format Specification}
\subsection{Reading JSON from Command-Line Argument}
The system allows users to specify an **input JSON file** as a command-line argument, ensuring flexibility and avoiding hardcoded file paths.

\subsection{JSON Input Format}
The system expects a structured JSON input, describing joints, motors, and arm segments.

\begin{verbatim}
{
    "robot_name": "ArmSpaceBot",
    "joints": [
        {
            "name": "shoulder",
            "position": [0, 0, 0],
            "motors": [
                {
                    "name": "shoulder_motor_x",
                    "rotation_axis": [1, 0, 0],
                    "ticks_per_revolution": 1000
                },
                {
                    "name": "shoulder_motor_z",
                    "rotation_axis": [0, 0, 1],
                    "ticks_per_revolution": 1200
                }
            ],
            "arm_segment": { "length": 10, "direction": [1, 0, 0] }
        },
        {
            "name": "elbow",
            "position": [10, 0, 0],
            "motors": [
                {
                    "name": "elbow_motor_y",
                    "rotation_axis": [0, 1, 0],
                    "ticks_per_revolution": 1100
                },
                {
                    "name": "elbow_motor_x",
                    "rotation_axis": [1, 0, 0],
                    "ticks_per_revolution": 900
                }
            ],
            "arm_segment": { "length": 10, "direction": [1, 0, 0] }
        }
    ],
    "desired_position": [10, 7, 5]
}
\end{verbatim}

\subsection{JSON Output Format}
The system calculates movements based on quaternion-based rotations and outputs motor tick values.

\begin{verbatim}
{
    "motor_ticks": [
        { "name": "shoulder_motor_x", "ticks": 97 },
        { "name": "shoulder_motor_z", "ticks": 82 },
        { "name": "elbow_motor_y", "ticks": 100 },
        { "name": "elbow_motor_x", "ticks": 105 }
    ],
    "status": "adjusted",
    "final_position": [10, 6.9, 4.9]
}
\end{verbatim}

\section{Error Handling}
\subsection{Error Reporting and Logging}
Standardized error codes ensure structured fault recovery.

\begin{table}[H]
\centering
\begin{tabular}{|c|l|p{8cm}|}
\hline
\textbf{Error Code} & \textbf{Message} & \textbf{Description} \\
\hline
ERR001 & Invalid Arm Length & A joint position does not match the expected length \& direction. \\
\hline
ERR002 & Disconnected Joint & A joint is not properly connected to the sequence. \\
\hline
ERR003 & Malformed JSON & The input structure is incorrect or missing required fields. JSON parsing failed. \\
\hline
ERR004 & Motor Data Missing & A motor definition is missing configurations, preventing tick calculation. \\
\hline
ERR005 & Unsupported Joint Count & The number of joints exceeds the maximum limit (future constraint: 6). \\
\hline
ERR006 & Invalid Rotation Axis & A motor has an incorrectly formatted rotation axis (should be a unit vector). \\
\hline
ERR007 & Quaternion Calculation Failure & Rotation computations failed due to incorrect vector inputs. \\
\hline
ERR008 & Tick Calculation Overflow & The computed tick value exceeds the allowed motor range. \\
\hline
\end{tabular}
\caption{Complete error code table for validation and computation}
\end{table}

\subsection{Validation Strategy and Behavior}
The \textbf{Validator} performs two levels of checks before computation:

\begin{enumerate}
	\item \textbf{Schema Validation}: The input JSON is checked against a strict JSON schema definition. This ensures all required fields, types, and formats are correct. If this validation fails, the validator terminates immediately with a schema error.
	
	\item \textbf{Content Consistency Validation}: If the schema check passes, the validator performs domain-specific semantic checks, including:
	\begin{itemize}
		\item All \textbf{rotation axes} are valid unit vectors.
		\item \textbf{Joint connectivity} is structurally correct (e.g., each joint's position corresponds to the preceding segment).
		\item Motor definitions are complete and values are within acceptable bounds.
		\item Arm segment directions and lengths are physically meaningful and consistent.
	\end{itemize}
\end{enumerate}

Unlike schema validation, content validation collects all issues and reports them together to aid debugging. These errors are returned in a structured JSON format that lists each violation with a corresponding error code and message.


\subsection{JSON Error Response Format}
If an error occurs, the system returns a structured JSON error message.

\begin{verbatim}
{
    "error": "Invalid arm length",
    "details": "Joint 'elbow' is not reachable from 'shoulder' based on defined segment length."
}
\end{verbatim}

\section{Mathematical Formulation}
\subsection{Quaternion-Based Rotation Computation}
To ensure precise, stable transformations and prevent **gimbal lock**, the system computes joint rotations using quaternions.

\subsection{Quaternion Representation}
A quaternion is defined as:



\[
q = w + x\mathbf{i} + y\mathbf{j} + z\mathbf{k}
\]



where:
\begin{itemize}
    \item \( w \) is the scalar (real) component.
    \item \( x, y, z \) are the vector (imaginary) components along **X, Y, Z axes**.
\end{itemize}

\subsection{Computing Rotation Using Quaternions}
Given a rotation by angle \( \theta \) around a normalized axis \( \mathbf{a} = (a_x, a_y, a_z) \), the quaternion is:



\[
q = \cos(\theta/2) + (a_x\mathbf{i} + a_y\mathbf{j} + a_z\mathbf{k}) \sin(\theta/2)
\]



For a given vector \( P = (p_x, p_y, p_z) \), the rotated vector \( P' \) is computed using:



\[
P' = q P q^{-1}
\]



where \( q^{-1} \) is the inverse quaternion:



\[
q^{-1} = \frac{w - x\mathbf{i} - y\mathbf{j} - z\mathbf{k}}{||q||^2}
\]



\section{Example Calculation: Quaternion-Based Arm Rotation}
\subsection{Robot Configuration}
\begin{itemize}
    \item \textbf{Joint 1 (Shoulder)}: Positioned at \( (0, 0, 0) \), rotates **around Z-axis**.
    \item \textbf{Joint 2 (Elbow)}: Positioned at \( (10, 0, 0) \), rotates **around Y-axis**.
    \item \textbf{Arm segments}:
        \begin{itemize}
            \item Segment **1**: Length = 10, direction = \([1, 0, 0]\).
            \item Segment **2**: Length = 10, direction = \([1, 0, 0]\).
        \end{itemize}
    \item \textbf{Desired end-effector position}: \( (10, 7, 5) \).
\end{itemize}

\subsection{Computing Rotation Using Quaternions}
\textbf{Step 1: Convert Position Vectors to Quaternions}

\begin{itemize}
    \item Current vector: \( v_1 = [10, 0, 0] \).
    \item Target vector: \( v_2 = [10, 7, 5] \).
\end{itemize}

\textbf{Step 2: Compute Rotation Axis}

Using the **cross product**:



\[
\mathbf{axis} = v_1 \times v_2 = [0, -50, 70]
\]



Normalized:



\[
\mathbf{axis} = [0, -0.58, 0.82]
\]



\textbf{Step 3: Compute Rotation Angle}

Using the **dot product**:



\[
\theta = \cos^{-1} \left( \frac{10 \times 10 + 0 \times 7 + 0 \times 5}{|v_1| |v_2|} \right) = \cos^{-1} (0.81) \approx 35^\circ
\]



\textbf{Step 4: Compute Rotation Quaternion}



\[
q = \cos(\theta/2) + \sin(\theta/2) \cdot (0, -0.58, 0.82)
\]





\[
q = [0.98, 0, -0.29, 0.41]
\]



\textbf{Step 5: Apply Quaternion Transformation}



\[
P' = q P q^{-1}
\]



After calculations, the new joint position is **adjusted to** \( P' = [10, 6.9, 4.9] \), moving the arm toward the target **without gimbal lock**.

\subsection{Motor Tick Conversion}
Using motor resolution:



\[
\mathrm{Ticks} = \left(\frac{\theta \times \text{ticks\_per\_revolution}}{360} \right)
\]



For **shoulder** (motor resolution: **1000** ticks/rev):



\[
\mathrm{Ticks}_{\text{shoulder}} = \left(\frac{35 \times 1000}{360} \right) = 97
\]



For **elbow** (motor resolution: **1200** ticks/rev):



\[
\mathrm{Ticks}_{\text{elbow}} = \left(\frac{30 \times 1200}{360} \right) = 100
\]



\subsection{Final JSON Output}
\begin{verbatim}
{
    "motor_ticks": [
        { "name": "shoulder_motor_x", "ticks": 97 },
        { "name": "shoulder_motor_z", "ticks": 82 },
        { "name": "elbow_motor_y", "ticks": 100 },
        { "name": "elbow_motor_x", "ticks": 105 }
    ],
    "status": "adjusted",
    "final_position": [10, 6.9, 4.9]
}
\end{verbatim}



\section{Functional Requirements}
\begin{itemize}
  \item FR1: The system shall parse structured JSON input files.
  \item FR2: The system shall perform computations using the parsed input.
  \item FR3: The system shall generate structured output in JSON format.
  \item FR4: The system shall validate input and report errors clearly.
\end{itemize}

\section{Non-Functional Requirements}
\begin{itemize}
  \item NFR1: The system shall run on macOS and Windows platforms.
  \item NFR2: The system shall be written in portable C++.
  \item NFR3: The system shall process typical input within 2 seconds.
  \item NFR4: The system shall achieve a minimum of 90\% unit test coverage.
\end{itemize}

\section{Testing Strategy}
\subsection{Overview}
Testing includes unit, regression, and basic integration tests to ensure correctness and stability.

\subsection{Unit Testing}
Implemented using GoogleTest (gtest), covering modules like the parser and calculation engine.

\subsection{Regression Testing}
Uses prepared input/output JSON files to compare results with reference data.

\subsection{Integration Testing}
Regression tests will serve as integration tests by validating module interactions.
\end{document}
